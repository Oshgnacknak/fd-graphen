\documentclass[a4paper, ngerman]{article}

\usepackage{babel}
\usepackage{csquotes}
\usepackage{amsmath}
\usepackage{tikz}

\usetikzlibrary{graphs}

\title{Funktionale Abhängigkeit und Graphen}
\author{Oshgnacknak}
\date{\today}

\setlength{\parindent}{0em}
\setlength{\parskip}{1em}

\begin{document}

\maketitle

Damit Datenbanksysteme Optimal arbeiten können,
müssen die in ihnen eingetragenen Daten
gewisse Eigenschaften erfüllen.
Für die Analyse dieser Eigenschaften
sind funktionale Abhängigkeiten essentiell.

Wir wollen uns in diesem Dokument damit befassen,
wie man funktionale Abhängigkeiten durch Graphen ausdrücken
und mit ihnen arbeiten kann.

Bevor wir beginnen noch
ein Wort zur Mengennotation:
Würde wir Mengen, wie sonst üblich schrieben,
kann es bei den Graphen echt unübersichtlich werden.
Daher wollen wir die Menge $\{ A_0, \dots, A_n \}$
als Zeichenkette abkürzen: $A_0 \dots A_n$.
Wenn wir die gesamte Menge meinen, 
schreiben wir auch $A_i$.

\section*{Funktionale Abhängigkeiten}
Eine funktionale Abhängigkeit
ist eine Relation zwischen zwei Attributmengen.
Nämlich die \enquote{folgt aus} Beziehung:
Wir schreiben
$$
    A_i \to B_i
$$
wenn in einer Tabelle (Relation) alle Zeilen (Tupel),
die in den Spalten (Attributen) $A_i$ übereinstimmen,
auch in den Spalten $B_i$ übereinstimmen.

Als Beispiel soll diese Tabelle dienen:
\begin{center}
\begin{tabular}{ c | c | c }
    Straße & PLZ & Ort \\
    \hline
    Heinertstraße & 37308 & Wingerode \\
    Hatter Straße & 37308 & Wingerode \\
    Hohenwestedter & 88499 & Emeringen \\
    Neubanz & 88500 & Emeringen \\
    Heinertstraße & 56283 & Beulich \\
\end{tabular}
\end{center}
Kennt man die PLZ, ist auch der Ort eindeutig.
$$
    \text{PLZ} \to \text{Ort}
$$
Kennt man wiederrum den Ort und die Straße,
kann man auf die PLZ zurück schließen.
$$
    \text{Straße}, \text{Ort} \to \text{PLZ}
$$

\section*{Eigenschaften funktionaler Abhängigkeit}
Um uns klar zu machen,
warum gerade Graphen eine gute Darstellung für
funktionale Abhängigkeiten sind,
wollen wir ein paar Eigenschaften zu
funktionaler Abhängigkeit herleiten.

\subsection*{Reflexivität}
Eine Relation ist reflexiv,
wenn jedes Element mit sich selbst in einer Beziehung steht.
Das ist bei funktionalen Abhängigkeiten natürlich der Fall,
denn wenn die Spalten $A_i$ festgelegt wurden,
sind insbesondere die Werte $A_i$ bestimmt.
$$
    A_i \to A_i
$$

\subsection*{Transitivität}
Eine Relation ist transitiv,
wenn sich die Relation zwischen mehreren Elementen überträgt.
Also wenn aus $A_i \to B_i$ und $B_i \to C_i$ schon $A_i \to C_i$ folgt.

Für funktionale Abhängigkeiten ist auch dies keine schwere Überlegung,
denn wenn die Werte $A_i$ feststehen,
folgen daraus die Werte für $B_i$,
wonach wir aber auch eindeutige Werte für $C_i$ bekommen.

Es ist uns somit erlaubt,
\enquote{$\to$} zu verketten,
was wir für Graphen unbedingt brauchen.
$$
    A_i \to B_i \to C_i
$$

\subsection*{Teilmengen}
Reflexivität ist nur ein Sonderfall hier von,
aber ich wollte es explizit erwähnen.
Die Idee ist, wenn wir die Werte $A_i$ festlegen,
und dann eine Teilmenge $B \subseteq A$ davon auswählen,
bleiben die Werte gleich.
$$
    A_i \to B_i
$$

Das erlaubt uns insbesondere,
funktionale Abhängigkeiten aufzuteilen,
sodass die rechte Seite
immer nur genau ein Attribut hat.
Anstatt $ABC \to DEF$ können wir also
$$
    ABC \to D,\;
    ABC \to E,\;
    ABC \to F
$$
schreiben, solange wir alle drei
funktionalen Abhängigkeiten hinschreiben.
Das ist auch wichtig, denn eine Kante
in einem Graphen hat immer genau einen Zielknoten.

\section*{Funktionale Abhängigkeiten als Graphen}
Zur einer gegebenen Menge an funktionalen
Abhängigkeiten können
wir uns nun endlich überlegen,
wie deren Graph auszusehen hat.

Zunächst bekommt jede Attributmenge
auf der Linken Seite einen Knoten,
der mit der Menge selbst beschriftet ist.
Sind z.B. die funktionalen Abhängigkeiten
$$
    AB \to C,\;
    CD \to E,\;
    EF \to AGH
$$
gegeben, brauchen wir die Knoten $AB$, $CD$ und $EF$.
\begin{center}
    \tikz{
        % \node[] (a) at (0,0) {$A$};
        \node[] (ab) at (1,0) {$AB$};
        % \node[] (b) at (1,-1) {$B$};
        % \node[] (c) at (1,-2) {$C$};
        \node[] (cd) at (1,-3) {$CD$};
        % \node[] (d) at (0,-3) {$D$};
        % \node[] (e) at (-1,-3) {$E$};
        \node[] (ef) at (-2,-3) {$EF$};
        % \node[] (f) at (-2,-2) {$F$};
        % \node[] (g) at (-2,-1) {$G$};
        % \node[] (h) at (-2,0) {$H$};

        \graph{
            % (ab) -> (a);
            % (ab) -> (b);
            % (ab) ->[bend left] (c);

            % (cd) -> (c);
            % (cd) -> (d);
            % (cd) ->[bend left] (e);

            % (ef) -> (e);
            % (ef) -> (f);
            % (ef) ->[bend left] (g);
            % (ef) ->[bend left] (h);
            % (ef) -> (a);
        }
    }
\end{center}
Für die rechten Seiten bekommt
jedes Attribute seinen eigenen Knoten.
Hier also $C$, $E$, $A$, $G$, und $H$.
Die Kanten ergeben sich eben aus
den funktionalen Abhängigkeiten.
\begin{center}
    \tikz{
        \node[] (a) at (0,0) {$A$};
        \node[] (ab) at (1,0) {$AB$};
        % \node[] (b) at (1,-1) {$B$};
        \node[] (c) at (1,-2) {$C$};
        \node[] (cd) at (1,-3) {$CD$};
        % \node[] (d) at (0,-3) {$D$};
        \node[] (e) at (-1,-3) {$E$};
        \node[] (ef) at (-2,-3) {$EF$};
        % \node[] (f) at (-2,-2) {$F$};
        \node[] (g) at (-2,-1) {$G$};
        \node[] (h) at (-2,0) {$H$};

        \graph{
            % (ab) -> (a);
            % (ab) -> (b);
            (ab) ->[bend left] (c);

            % (cd) -> (c);
            % (cd) -> (d);
            (cd) ->[bend left] (e);

            % (ef) -> (e);
            % (ef) -> (f);
            (ef) ->[bend left] (g);
            (ef) ->[bend left] (h);
            (ef) -> (a);
        }
    }
\end{center}
Wir sind aber noch nicht fertig.
Wie wir uns überlegt haben,
können wir die Attributmenge
auf der linken Seite mit in die rechte übernehmen.
$$
    A_i \to B_i \implies
    A_i \to A_i, B_i 
$$

Diese Kanten sind im späteren Verlauf
noch wichtig und wir wollen sie auch gleich einzeichnen.
\begin{center}
    \tikz{
        \node[] (a)  at (0,0)   {$A$};
        \node[] (ab) at (1,0)   {$AB$};
        \node[] (b)  at (1,-1)  {$B$};
        \node[] (c)  at (1,-2)  {$C$};
        \node[] (cd) at (1,-3)  {$CD$};
        \node[] (d)  at (0,-3)  {$D$};
        \node[] (e)  at (-1,-3) {$E$};
        \node[] (ef) at (-2,-3) {$EF$};
        \node[] (f)  at (-2,-2) {$F$};
        \node[] (g)  at (-2,-1) {$G$};
        \node[] (h)  at (-2,0)  {$H$};

        \graph{
            (ab) -> (a);
            (ab) -> (b);
            (ab) ->[bend left] (c);

            (cd) -> (c);
            (cd) -> (d);
            (cd) ->[bend left] (e);

            (ef) -> (e);
            (ef) -> (f);
            (ef) ->[bend left] (g);
            (ef) ->[bend left] (h);
            (ef) -> (a);
        }
    }
\end{center}

\subsection*{Erreichbarkeit und die Hülle}
Bei funktionalen Abhängigkeiten ist
der Begriff der Attributhülle 
von großer Bedeutung.
Die Attributhülle von eine
gegebenen Attributmenge $A_i$
ist die Menge aller Attibute, 
die funktional von $A_i$ abhängen.
Wir schreiben $A_i^+$ für die Attributhülle von $A_i$.

Wie wir sehen werden,
sind das alle Knoten in unserem Graphen,
die von $A_i$ aus erreichbar sind,
denn auch bei dieser Fragestellung
gilt Transitivität und die Teilmengeneigenschaft.

Wir können uns somit für die Berechnung der Hülle
vollständig auf den Graphen konzentrieren.

Nehmen wir als Beispiel
die funktionalen Abhängigkeiten.
$$
    A  \to B,\;
    AB \to C,\;
    CD \to E,\;
    EF \to AGH
$$
und bestimmen die Hülle $EF^+$ von $EF$.
Hier der Graph:
\begin{center}
    \tikz{
        \node[] (a)  at (0,0)   {$A$};
        \node[] (ab) at (1,0)   {$AB$};
        \node[] (b)  at (1,-1)  {$B$};
        \node[] (c)  at (1,-2)  {$C$};
        \node[] (cd) at (1,-3)  {$CD$};
        \node[] (d)  at (0,-3)  {$D$};
        \node[] (e)  at (-1,-3) {$E$};
        \node[] (ef) at (-2,-3) {$EF$};
        \node[] (f)  at (-2,-2) {$F$};
        \node[] (g)  at (-2,-1) {$G$};
        \node[] (h)  at (-2,0)  {$H$};

        \graph{
            (a)  -> (b);
            (ab) -> (a);
            (ab) -> (b);
            (ab) ->[bend left] (c);

            (cd) -> (c);
            (cd) -> (d);
            (cd) ->[bend left] (e);

            (ef) -> (e);
            (ef) -> (f);
            (ef) ->[bend left] (g);
            (ef) ->[bend left] (h);
            (ef) -> (a);
        }
    }
\end{center}
Zunächst gilt Reflexivität und damit
$$
    EF \subseteq EF^+
$$
In anderen Worten: $EF$ ist von sich selber aus erreichbar.

Dann können wir alle Knoten hinzunehmen,
die direkt von $EF$ aus erreichbar sind.
$$
    AGH \subseteq EF^+  
$$
Es kommen nur $AGH$ dazu.

Vom Knoten $EF$ aus können wir nichts mehr lernen,
aber wir haben festgestellt, ist $EF$ bestimmt,
dann ist auch $A$ festgelegt.
Wir können die Berechnung also mit $A$ fortsetzen.
$$
    EF^+ = EF^+ \cup A^+
$$
Hier kommt die Transitivität
von funktionalen Abhängigkeiten zu trage.

Durch $A^+$ kommt sogleich der Knoten $B$ hinzu.
$$
    B \subseteq EF^+
$$
Wir nehmen Zwischenstand und es gilt
$$
    ABEFGH \subseteq EF^+
$$
Das gibt uns, dass wenn $EF$ bestimmt ist,
auch steht $AB$ fest.
Auch wenn hierzu noch keine Kante eingezeichnet ist,
haben wir dennoch gezeigt, dass es diese gibt
und wir können mit $AB^+$ weiter machen.
$$
    EF^+ = EF^+ \cup A^+ \cup AB^+
$$
Es kommst aber nur noch $C$ dazu
und wir habe die Hülle bestimmt.
$$
    EF^+ = ABCEFGH
$$

\section*{Die tatsächliche Größe des Graphen}
Wie gerade gesehen, haben wir nicht alle Kanten angegeben.
Tatsächlich haben wir auch nicht alle Knoten hingemalt,
denn z.B. der Knoten $ACE$ fehlt.
Wir müssten auch Folgendes
in unseren Graphen beifügen.
\begin{center}
\tikz{
    \node[] (ace) at (0,0)   {$ACE$};

    \node[] (ac) at (0,1)   {$AC$};
    \node[] (ae) at (-1,-1) {$AE$};
    \node[] (ce) at (1,-1)  {$CE$};

    \node[] (a) at (-1,0)  {$A$};
    \node[] (c) at (1,0)   {$C$};
    \node[] (e) at (0,-1)  {$E$};

    \graph{
        (ace) -> (ac);
        (ace) -> (ae);
        (ace) -> (ce);

        (ace) -> (a);
        (ace) -> (c);
        (ace) -> (e);

        (ac) -> (a);
        (ac) -> (c);

        (ae) -> (a);
        (ae) -> (e);

        (ce) -> (c);
        (ce) -> (e);
    }
}
\end{center}
Das würde aber sehr schnell unübersichtlich
und deswegen verzichten wir auf
ein Großteil dieser Informationen.

Zu unserem Nachteil ist ist dies aber nicht,
denn wenn wir später Normalisieren wollen,
werden wir feststellen,
dass wir tatsächlich noch mehr Knoten
und Kanten aus unseren Graphen verbannen wollen.

Man mache sich klar, dass es zu einer 
gegebenen Attributsmenge mindestens
$2^n$ Knoten geben muss.
Es gibt noch weitaus mehr Kanten.

\section*{Schlüssel}
Keine Datenbank kommt ohne Schlüssel aus.
Schlüssel sind dafür relevant,
Zeilen in unseren Tabellen eindeutig zu bestimmen.

Daher die Definition:
Eine Attributmenge ist ein Schlüssel,
gdw. alle Attribute in ihrer Hülle sind.
An anderen Worten: Sind die Schlüsselwerte bestimmt,
stehen alle Werte fest.

In unseren Graphen heißt das,
Schlüssel sind alle Knoten,
von denen aus der gesamte Graph erreichbar ist.

Manche nennen Schlüssel auch \enquote{Superschlüssel}.
Ich halte diesen Begriff für missverständlich,
da dies noch die allgemeinste Form
von Schlüsseln sind.

Ein Schlüssel heißt einfach,
gdw. er nur ein Element hat.
Z.B. ist im folgenden Graphen $A$ einfach,
$AB$ aber nicht.
\begin{center}
\tikz\graph[clockwise=4, radius=1cm]{
    {"$AB$", "$A$", "$B$", "$C$"};
    "$AB$" -> {"$A$" -> "$C$"} -> "$B$";
};
\end{center}
Ist ein Schlüssel nicht einfach,
ist er zusammengesetzt.
Im obigen Beispiel also der Schlüssel $AB$.

Ein Schüssel ist minimal,
wenn es keinen kleineren Schlüssel
im Sinne der Teilmengenordnung
\enquote{$\subseteq$} gibt.
Z.B ist hier der Schlüssel $AB$ minimal,
$ABC$ aber nicht.
\begin{center}
\tikz\graph[clockwise=4, radius=1cm]{
    {"$ABC$", "$AB$", "$C$", "$D$"};
    "$ABC$" -> "$AB$";
    "$AB$" -> {"$C$", "$D$"};
};
\end{center}
Ein Schlüssel ist trivial,
gdw. er alle Attribute in der Relation beinhaltet.
Im obigen Beispiel wäre also $ABCD$ der triviale Schlüssel.

Minimale Schlüssel sind von besonderer Bedeutung.
Daher erhalten sie eine eigenen Namen: \enquote{Schlüsselkandidat}.
Diese Bezeichnung halte ich wieder für etwas widersprüchlich,
da Schlüsselkandidaten schon Schlüssel sind -
sie müssen nicht dafür kandidieren.

Es ist daher hilfreich, sich klar zu machen,
dass \enquote{Schlüsselkandidat} wiederrum kurz
für \enquote{Primärschlüsselkandidat} ist.
Ein \enquote{Primärschlüssel},
ist ein von uns ausgewählter Schlüsselkandidat.

Hier nochmal die Hierarchie in einer Liste:
\begin{enumerate}
    \item Normale Attributmenge
    \item Schlüssel: Attributmenge mit maximaler Hülle
    \item Schlüsselkandidat: minimaler Schlüssel ($\subseteq$)
    \item Primärschlüssel: Speziell ausgewählter Schlüsselkandidat
\end{enumerate}

Jedes Attribut,
das Teil eines beliebigen Schlüsselkandidaten ist,
nennen wir \enquote{Primattribut}.
Alle anderen nennen wir dementsprechend
\enquote{Nichtprimattribute}.

\section*{Bestimmen der Schlüsselkandidaten}
Wir wollen uns an folgenden Beispiel anschauen,
wie man Schlüsselkandidaten aus dem Graphen zu eine
funktionale Abhängigkeitsmenge auslesen kann.
Gegeben also die funktionalen Abhängigkeiten:
$$
    A  \to C,\;
    B  \to DE,\;
    C  \to F,\;
    F  \to G,\;
    FG \to A,\;
    E  \to G,\;
    A  \to F
$$
Und deren Graph:
\begin{center}
\tikz\graph[clockwise=8, radius=2cm]{
    {"$A$", "$C$", "$F$", "$E$", "$B$", "$D$", "$G$", "$FG$"};
    "$A$" -> {"$C$", "$F$"};
    "$B$" -> {"$E$", "$D$"};
    "$C$" -> "$F$";
    "$E$" -> "$G$";
    "$F$" -> "$G$";
    "$FG$" -> {"$F$", "$G$", "$A$"};
};
\end{center}
Bevor wir uns weitere Gedanken machen,
sollte uns zwei Dinge beim betrachten des Graphen auffallen:
\begin{enumerate}
    \item Der Knoten $B$ kann von keinem
        anderen Knoten aus erreicht werden.
        Damit muss er Teil jedes Schlüsselkandidaten sein.
    \item Die Knoten $D$ und $G$ haben
        keine ausgehenden Kanten.
        Sie sind für die Schlüsselkandidaten uninteressant.
\end{enumerate}
Wir wollen die Hülle aller noch relevanten Knoten berechnen:
\begin{align*}
    A^+ &= ACFG \\
    B^+ &= BDEG \\
    C^+ &= ACFG \\
    E^+ &= EG \\
    F^+ &= ACFG
\end{align*}
$A$, $C$ und $F$ haben die gleiche Hülle
und sind damit austauschbar.
Wir nehmen $B$ hinzu und haben
jeweils alles abgedeckt.
Es ergeben sich die Schlüsselkandidaten:
$$
    AB, CB, FB
$$

\section*{Die drei Normalformen}
\emph{
    The key, the whole key, and nothing but the key.
    So help me Codd!}

Dieser Spruch soll dabei unterschützen,
sich die ersten drei Normalformen zu merken.
Wir werden uns diese im Folgenden anschauen.
Wir werden uns auch überlegen,
welche Eigenschaften die Graphen haben müssen
um eine Normalform zu erfüllen.

\subsection*{1. Normalform: The key}
\emph{Eine Relation ist in 1. Normalform,
    wenn alle Attributes Atomar sind.}

Damit man überhaupt funktionale Abhängigkeiten
und darüber Schlüssel
(daher \enquote{The Key}) bestimmen kann,
darf in einer Spalte immer nur genau ein Wert stehen. 

Da wir uns hauptsächlich mit
funktionalen Abhängigkeiten beschäftigen,
gehen wir im weiteren Verlauf davon aus,
dass alle Relation in 1. Normalform sind.

\subsection*{2. Normalform: The whole key}
\emph{Eine Relation ist in 2. Normalform,
    wenn sie in 1. Normalform ist
    und alle Nichtprimattribute vollständig
    von jedem Schlüsselkandidat abhängen.}

Wir müssen hierfür den Begriff der
vollständigen Abhängigkeit definieren.
Ein Attribut ist genau dann vollständigen
von einer Attributmenge abhängen,
wenn man kein Attribut aus der Attributmenge entfernen kann,
ohne die funktionale Abhängigkeit zu verlieren.

Im folgenden Beispiel ist $C$ vollständig von $AB$ abhängig:
\begin{center}
\tikz\graph[clockwise=4, radius=1cm]{
    {"$A$", "$AB$", "$C$", "$D$"};
    "$A$" <- "$AB$" -> {"$C$", "$D$"};
    "$A$" -> "$D$"
};
\end{center}
$D$ hängt schon von $A$ ab
und ist damit nur partiell von $AB$ abhängig.

\subsection*{3. Normalform: And nothing but the key}
\emph{Eine Relation ist in 3. Normalform,
    wenn sie in 2. Normalform ist
    und alle Nichtprimattribute nicht transitiv
    von einem Schlüsselkandidat abhängen.}

Den Begriff der transitiven Abhängigkeit
haben wird grundsätzlich schon oben kennen gelernt.
Es gibt aber noch ein paar Details zu klären.
Zur Erinnerung: $B$ ist hier direkt von $A$ abhängig:
$$
    A \to B \to C
$$
$C$ ist transitiv von $A$, über $B$, abhängig.

Wichtig ist zusätzlich, dass eben nicht $B \to A$ gilt.
Hier also nochmal die volle Definition:
Eine Attributmenge $C_i$ ist genau dann transitiv
von eine Attributmenge $A_i$ über $B_i$ abhängig, wenn
$$
    A_i \to B_i \to C_i
$$
aber nicht 
$$
    B_i \to A_i
$$
gilt.

\subsection*{Beispiele}

Die folgenden funktionalen Abhängigkeiten
sind nur in 1. Normalform:
\begin{center}
\tikz\graph[clockwise=3, radius=1cm]{
    {"$AB$", "$B$", "$A$"};
    "$AB$" -> {"$B$", "$A$"};
    "$A$" -> "$B$"
};
\end{center}
$B$ ist partiell von $AB$ abhängig.

Hier ist eine funktionale Abhängigkeitsmenge
in 2. Normalform:
\begin{center}
\tikz\graph[clockwise=5, radius=1cm]{
    {"$A$", "$B$", "$C$", "$D$", "$E$"};
    "$A$" -> {"$B$", "$C$", "$D$" -> "$E$"};
};
\end{center}
Es gibt keine partiellen Abhängigkeiten,
aber $E$ ist transitiv von $A$ abhängig.

Und nun funktionale Abhängigkeiten in 3. Normalform:
\begin{center}
\tikz\graph[clockwise=5, radius=1cm]{
    {"$A$", "$B$", "$C$", "$D$", "$E$"};
    "$A$" -> {"$B$", "$C$" -> "$D$", "$E$"};
    "$C$" ->[bend left] "$A$"
};
\end{center}
Es gibt keine partiellen Abhängigkeiten
und $D$ ist nicht transitiv abhängig von $A$,
da auch $C \to A$ gilt.

\section*{Normalisieren}
Wir wollen uns nun anschauen,
wie man die jede funktionale Abhängigkeitsmenge
in die 3. Normalform bringen kann.

\subsection*{Löschen überflüssiger Kanten}

Gegeben sei folgender Graph,
den wir normalisieren sollen:
\begin{center}
\tikz\graph[clockwise=7, radius=2cm]{
    {"$A$", "$AB$", "$B$", "$C$", "$D$", "$E$", "$F$"};
    "$AB$" -> {"$A$", "$B$", "$E$", "$C$", "$F$"};
    "$A$" -> "$F$";
    "$AB$" -> {"$F$", "$D$"};
    "$E$" -> "$D$";
};
\end{center}
Die 2. Normalform ist nicht erfüllt,
da $F$ partiell von $AB$ abhängt.
Man könnte das auch wie folgt auffassen:
Die Kante $AB \to F$ tragt
nichts zur Hülle von $AB$ bei und ist damit überflüssig.
Genau so ist die Kante $AB \to D$
nicht von Bedeutung.

Wir entfernen die Kanten und die 2.
Normalform ist erfüllt:
\begin{center}
\tikz\graph[clockwise=7, radius=2cm]{
    {"$A$", "$AB$", "$B$", "$C$", "$D$", "$E$", "$F$"};
    "$AB$" -> {"$A$", "$B$", "$E$", "$C$"};
    "$A$" -> "$F$";
    "$E$" -> "$D$";
};
\end{center}

\subsection*{Aufteilen transitiver Abhängigkeiten}

Hier noch ein Beispiel:
\begin{center}
\tikz\graph[clockwise=4, radius=1cm]{
    {"$A$", "$B$" -> "$E$", "$AB$", "$C$" -> "$D$", "$D$", "$E$"};
    "$AB$" -> {"$A$", "$B$", "$C$"};
    "$A$" -> "$B$";
};
\end{center}
$B$ ist sowohl direkt, als auch partiell von $AB$ abhängig.
Wir löschen also die direkte Kante:
\begin{center}
\tikz\graph[clockwise=4, radius=1cm]{
    {"$A$", "$B$" -> "$E$", "$AB$", "$C$" -> "$D$", "$D$", "$E$"};
    "$AB$" -> {"$A$", "$C$"};
    "$A$" -> "$B$";
};
\end{center}
Wir überlegen uns die Hülle von $A$.
Offensichtlich gilt $AB \subseteq A^+$
und über $AB$ und $B$ dann auch $A^+ = ABCDE$.
Wir wollen die fehleden Kanten
von $A$ nach $C$ und $AB$ temporär einzeichnen.
\begin{center}
\tikz\graph[clockwise=4, radius=1cm]{
    {"$A$", "$B$" -> "$E$", "$AB$", "$C$" -> "$D$", "$D$", "$E$"};
    "$AB$" -> {"$A$", "$C$"};
    "$A$" -> {"$B$", "$C$"};
    "$A$" ->[bend left] "$AB$";
};
\end{center}
Wir haben damit aber zugleich eine partiell
und eine transitive Abhängigkeiten erzeugt.
Beides beheben wir, indem wir die Kante $AB \to C$ verbannen: 
\begin{center}
\tikz\graph[clockwise=4, radius=1cm]{
    {"$A$", "$B$" -> "$E$", "$AB$", "$C$" -> "$D$", "$D$", "$E$"};
    "$AB$" -> "$A$";
    "$A$" -> {"$B$", "$C$"};
    "$A$" ->[bend left] "$AB$";
};
\end{center}
Man beachte, dass sich die Hülle jedes Knoten
in den obigen Schritten nicht geändert hat.

Der Knoten $AB$ ist überflüssig geworden und kann entfernt werden.
Dies ist erlaubt, da er zusammengesetzt ist
und sich damit die Attributmenge nicht ändert. 
\begin{center}
\tikz\graph{
    "$D$" <- "$C$" <- "$A$" -> "$B$" -> "$E$";
};
\end{center}
Man könnte sagen, der Knoten $A$ hat den Knoten $AB$ verschluckt.
Wir haben die 2. Normalform hergestellt.

Nun zum eigentlichen Thema:
Wir wollen die transitiven Abhängigkeiten los werden.
Wie können wir da vor gehen?
Jede Kante ist essentiell und
weitere Kanten machen es nur wieder schlechter.

Der Trick ist, das wir hier zum
ersten mal unseren Graphen aufteilen.
Und zwar nicht nur den:
Tatsächlich können wir jetzt ablesen,
in welche Tabellen wir unsere ursprüngliche Relation
aufteilen müssen, sodass jede dieser die 3. Normalform erfüllt.
Wir teilen immer bei den transitiven Abhängigkeiten auf:
\begin{center}
\tikz\graph{
    "$C'$" -> "$D$";
    "$C$" <- "$A$" -> "$B$";
    "$B'$" -> "$E$";
};
\end{center}
Gemeint ist, dass die Attribute $B'$ und $C'$ jeweils Kopien 
in einer anderen Tabelle sind.
Wir haben also aus der einen Relation $ABCDE$,
die drei Relationen $C'D$, $ABC$ und $B'E$ gemacht,
sodass jede Tabelle die 3. Normalform erfüllt.

\subsection*{Der Synthesealgorithmus}
Wir haben uns jetzt schon abgeschaut,
wie man intuitiv eine Relation normalisieren kann.
Der Synthesealgorithmus soll uns dabei helfen,
einige Sonderfälle richtig zu behandeln.

Man beachte, dass der Synthesealgorithmus nicht
für Graphen entworfen wurde,
sondern direkt auf den
funktionalen Abhängigkeiten arbeitet.
Wir haben daher einige Schritte zusammengefasst,
während wir andere weiter unterteilt haben.

Hier also die Schritte des Synthesealgorithmus:
\begin{enumerate}
\item Linksreduktion:
    Ziel ist, die linke Seite einer
    funktionalen Abhängigkeit zu vereinfachen.
    Z.B. wird aus
    $$
        AC \to B,\;
        A \to D,\;
        D \to BC
    $$
    indem man das $C$ auf der linken Seite streicht
    $$
        A \to B,\;
        A \to D,\;
        D \to BC
    $$
    Das ist genau das \enquote{Verschlucken},
    das wir zuvor entdeckt habe. Aus
    \begin{center}
    \tikz\graph[clockwise=5, radius=1cm]{
        {"$A$", "$AC$", "$C$", "$B$", "$D$"};
        "$AC$" -> {"$A$", "$B$", "$C$"};
        "$A$" -> "$D$";
        "$D$" -> {"$B$", "$C$"};
    };
    \end{center}
    wurde also
    \begin{center}
    \tikz\graph[clockwise=4, radius=1cm]{
        {"$A$", "$B$", "$C$", "$D$"};
        "$A$" -> {"$D$", "$B$"};
        "$D$" -> {"$B$", "$C$"};
    };
    \end{center}

\item Rechtsreduktion:
    Hier soll die rechte Seite minimiert werden.
    Ist z.B.
    $$
        A \to BC,\;
        B \to D,\;
        D \to C
    $$
    gegeben, kann das $C$ in der
    ersten funktionalen Abhängigkeit
    weggelassen werden.
    $$
        A \to B,\;
        B \to D,\;
        D \to C
    $$
    Dieser Schritt ist analog zum
    Entfernen der überflüssigen Kanten. Aus
    \begin{center}
    \tikz\graph[clockwise=4, radius=1cm]{
        {"$A$", "$B$", "$D$", "$C$"};
        "$A$" -> {"$B$", "$C$"};
        "$B$" -> "$D$";
        "$D$" -> "$C$";
    };
    \end{center}
    wurde hierbei
    \begin{center}
    \tikz\graph{
        "$A$" -> "$B$" -> "$D$" -> "$C$";
    };
    \end{center}

\item Entfernen von $A_i \to \emptyset$:
    Dieser Schritt entfällt,
    da wir es in unseren Graphen sowieso machen.

\item Zusammenfassung von funktionalen Abhängigkeiten
    mit gleichen linken Seiten:
    Dieser Schritt entfällt auch.
    In unseren Graphen gibt es
    jede linke Seite nur ein mal.

\item Falls nötig, hinzufügen eines Schlüsselknoten:
    Gehen wir davon aus,
    dass wir durch die vorherigen Schritte
    diesen Graphen erhalten haben:
    \begin{center}
    \tikz\graph[clockwise=5, radius=1cm]{
        {"$A$", "$E$" -> "$G$", "$B$" -> "$D$", "$C$", "$F$"};
        "$A$" -> {"$C$", "$E$"};
        "$B$" -> {"$D$", "$E$"};
        "$C$" -> "$F$";
        "$E$" -> "$G$";
        "$F$" -> "$A$";
    };
    \end{center}
    Problem ist, dass es keinen Knoten gibt,
    der als Primärschlüssel dienen kann.\

    Wir müssen also einen neuen Knoten hinzufügen,
    der unser Primärschlüssel sein wird.
    Ein möglicher Schlüsselkandidat ist $BC$, wir nehmen in.
    \begin{center}
    \tikz\graph[clockwise=6, radius=1cm]{
        {"$A$", "$E$" -> "$G$", "$B$" -> "$D$", "$BC$", "$C$", "$F$"};
        "$A$" -> {"$C$", "$E$"};
        "$B$" -> {"$D$", "$E$"};
        "$C$" -> "$F$";
        "$E$" -> "$G$";
        "$F$" -> "$A$";
        "$BC$" -> {"$B$", "$C$"};
    };
    \end{center}
    Ist vor diesem Schritt immer noch
    ein Schlüsselkandidat im Graph enthalten,
    können wir diesen Schritt ganz überspringen.

\item Fasse zyklische Knoten zusammen:
    Nehmen wir uns diesen Graph noch mal vor
    \begin{center}
    \tikz\graph[clockwise=6, radius=1cm]{
        {"$A$", "$E$" -> "$G$", "$B$" -> "$D$", "$BC$", "$C$", "$F$"};
        "$A$" -> {"$C$", "$E$"};
        "$B$" -> {"$D$", "$E$"};
        "$C$" -> "$F$";
        "$E$" -> "$G$";
        "$F$" -> "$A$";
        "$BC$" -> {"$B$", "$C$"};
    };
    \end{center}
    Was uns ins Auge fallen sollte,
    ist der Zyklus $ACF$.\

    Transitivität gibt uns,
    dass z.B. nicht nur durch $A$
    die Werte von $C$ bestimmt sind,
    sondern auch durch $C$ die Werte von $A$.
    Damit können Werte von $A$ und $C$ immer
    nur in Paaren auftauchen. \
    
    Ähnliche Argumente gelten für $AF$ und $CF$.
    Wir können diese drei Knoten damit
    als einen Großen betrachten.
    \begin{center}
    \tikz\graph[clockwise=5, radius=1cm]{
        {"$ACF$", "$E$" -> "$G$", "$B$", "$D$", "$BC$"};
        "$ACF$" -> "$E$";
        "$B$" -> {"$D$", "$E$"};
        "$E$" -> "$G$";
        "$BC$" -> {"$B$", "$ACF$"};
    };
    \end{center}
    Dieser Schritt ist äquivalent zum Schritt
    \enquote{Fasse Relationen zusammen}
    des Synthesealgorithmus.

\item Erzeuge eine Relation pro funktionalen Abhängigkeit:
    Das ist der letzte Schritt beim
    Aufteilen transitiver Abhängigkeiten.
    Jeder Knoten, der ausgehende Kanten hat,
    bekommst eine funktionalen Abhängigkeit
    mit seinen alles direkten Nachbarn
    auf der rechten Seite. 
    Aus dem Graphen
    \begin{center}
    \tikz\graph{
        "$A$" <- "$B$" -> "$C$" -> "$D$";
    };
    \end{center}
    erhalten wir beispielsweise
    die funktionalen Abhängigkeiten
    $$
        B \to AC,\;
        C' \to D
    $$
    Jede funktionale Abhängigkeit gibt uns dann genau eine Relation.
    Wir erhalten also die zwei Relationen
    $$
        ABC,\; C'D
    $$
    Wieder ist $C'$ als Kopie von $C$ zu verstehen.
    Die hier entstehenden Relationen sind in der 3. Normalform.
\end{enumerate}

\end{document}
