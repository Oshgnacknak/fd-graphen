\documentclass[a4paper, ngerman]{article}

\usepackage{babel}
\usepackage{csquotes}
\usepackage{amsmath}

\title{Funktionelle Abhängigkeit und Graphen}
\author{Oshgnacknak}
\date{\today}

\begin{document}

\maketitle

Damit Datenbanksysteme Optimal arbeiten können,
müssen die in ihnen eingetragenen Daten
gewisse Eigenschaften erfüllen.
Für die Analyse dieser Eigenschaften
Sind funktionelle Abhängigkeiten essentiell.

\section*{Funktionelle Abhängigkeiten}
Eine funktionelle Abhängigkeit
(im Folgenden nur noch \enquote{FD}, für \emph{functional dependency})
ist eine Relation zwischen zwei Attributmengen.
Nämlich die \enquote{folgt aus} Beziehung:
Wir schreiben
$$
    A_0, \dots, A_n \to 
    B_0, \dots, B_m
$$
wenn in einer Tabelle (Relation) alle Zeilen (Tupel),
die in den Spalten (Attributen) $A_i$ übereinstimmen
auch in den Spalten $B_i$ übereinstimmen.

Als Beispiel soll folgende Tabelle dienen:
\begin{center}
\begin{tabular}{ c | c | c }
    Straße & PLZ & Ort \\
    \hline
    Heinertstraße & 37308 & Wingerode \\
    Hatter Straße & 37308 & Wingerode \\
    Hohenwestedter & 88499 & Emeringen \\
    Neubanz & 88500 & Emeringen \\
    Heinertstraße & 56283 & Beulich \\
\end{tabular}
\end{center}
Kennt man die PLZ, ist auch der Ort eindeutig
$$
    \text{PLZ} \to \text{Ort}
$$
Kennt man wiederrum den Ort und die Straße,
kann man auf die PLZ zurück schließen
$$
    \text{Straße}, \text{Ort} \to \text{PLZ}
$$

\section*{Eigenschaften funktioneller Abhängigkeit}
Um uns klar zu machen,
warum gerade Graphen eine gute Darstellung für
funktionelle Abhängigkeiten sind,
wollen wir ein paar Überlegungen zu FDs machen.

\subsection*{Reflexivität}
Eine Relation ist reflexiv,
wenn jeder Wert mit sich selbst in einer Beziehung steht.
Das ist bei FDs natürlich der Fall,
denn wenn die Spalten $A_i$ festgelegt wurden,
sind insbesondere die Werte $A_i$ bestimmt.
$$
    A_i \to A_i
$$
\end{document}
