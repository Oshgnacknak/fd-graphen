\documentclass[a4paper, ngerman]{article}

\usepackage{babel}
\usepackage{csquotes}
\usepackage{amsmath}
\usepackage{tikz}

\usetikzlibrary{graphs}

\title{Funktionelle Abhängigkeit und Graphen}
\author{Oshgnacknak}
\date{\today}

\begin{document}

\maketitle

Damit Datenbanksysteme Optimal arbeiten können,
müssen die in ihnen eingetragenen Daten
gewisse Eigenschaften erfüllen.
Für die Analyse dieser Eigenschaften
Sind funktionelle Abhängigkeiten essentiell.

Wir wollen uns im weiteren Verlauf dieses Dokuments damit befassen,
wie man funktionelle Abhängigkeiten durch Graphen ausdrücken
und mit ihnen arbeiten kann.

\section*{Funktionelle Abhängigkeiten}
Eine funktionelle Abhängigkeit
(im Folgenden nur noch \enquote{FD}, für \emph{functional dependency})
ist eine Relation zwischen zwei Attributmengen.
Nämlich die \enquote{folgt aus} Beziehung:
Wir schreiben
$$
    A_0, \dots, A_n \to 
    B_0, \dots, B_m
$$
wenn in einer Tabelle (Relation) alle Zeilen (Tupel),
die in den Spalten (Attributen) $A_i$ übereinstimmen
auch in den Spalten $B_i$ übereinstimmen.

Als Beispiel soll folgende Tabelle dienen:
\begin{center}
\begin{tabular}{ c | c | c }
    Straße & PLZ & Ort \\
    \hline
    Heinertstraße & 37308 & Wingerode \\
    Hatter Straße & 37308 & Wingerode \\
    Hohenwestedter & 88499 & Emeringen \\
    Neubanz & 88500 & Emeringen \\
    Heinertstraße & 56283 & Beulich \\
\end{tabular}
\end{center}
Kennt man die PLZ, ist auch der Ort eindeutig
$$
    \text{PLZ} \to \text{Ort}
$$
Kennt man wiederrum den Ort und die Straße,
kann man auf die PLZ zurück schließen
$$
    \text{Straße}, \text{Ort} \to \text{PLZ}
$$

\section*{Eigenschaften funktioneller Abhängigkeit}
Um uns klar zu machen,
warum gerade Graphen eine gute Darstellung für
funktionelle Abhängigkeiten sind,
wollen wir ein paar Überlegungen zu FDs machen.

\subsection*{Reflexivität}
Eine Relation ist reflexiv,
wenn jeder Wert mit sich selbst in einer Beziehung steht.
Das ist bei FDs natürlich der Fall,
denn wenn die Spalten $A_i$ festgelegt wurden,
sind insbesondere die Werte $A_i$ bestimmt.
$$
    A_i \to A_i
$$

\subsection*{Transitivität}
Eine Relation ist transitiv,
wenn sich die Relation zwischen mehreren Werten überträgt.
Also wenn aus $A_i \to B_i$ und $B_i \to C_i$ schon $A_i \to C_i$ folgt.

Für FDs ist auch dies keine schwere Überlegung,
denn wenn die Werte $A_i$ feststehen,
folgen daraus die Werte für $B_i$,
wonach wir aber auch eindeutige Werte für $C_i$ bekommen.

Es ist uns somit erlaubt,
\enquote{$\to$} zu verketten,
was wir für Graphen unbedingt brauchen
$$
    A_i \to B_i \to C_i
$$

\subsection*{Teilmengen}
Reflexivität ist nur ein Sonderfall hier von,
aber ich wollte es explizit erwähnen.
Die Idee ist, wenn wir die Werte $A_i$ festlegen,
und dann eine Teilmenge $B \subseteq A$ davon auswählen.
Bleiben die Werte gleich.
$$
    A_i \to B_i
$$

Das erlaubt uns insbesondere,
FDs aufzuteilen, sodass die rechte Seite
immer nur genau ein Element hat.
Anstatt $ABC \to DEF$ können wir also
$$
    ABC \to D,\;
    ABC \to E,\;
    ABC \to F
$$
schreiben, solange wir alle drei FDs hinschreiben.
Das ist auch wichtig, denn eine Kante
in einem Graph hat immer genau einen Zielknoten.

\section*{Funktionelle Abhängigkeiten als Graphen}
Zur einer gegebenen Menge an FDs können
wir uns nun endlich überlegen,
wie deren Graph aus zu stehen hat.

Zunächst bekommt jede Attributmenge
auf der Linken Seite einen Knoten,
der mit der Menge selbst beschriftet ist.
Sind z.B. die FDs
$$
    AB \to C,\;
    CD \to E,\;
    EF \to AGH
$$
gegeben, brauchen die Knoten $AB$, $CD$ und $EF$.
\begin{center}
    \tikz{
        % \node[] (a) at (0,0) {$A$};
        \node[] (ab) at (1,0) {$AB$};
        % \node[] (b) at (1,-1) {$B$};
        % \node[] (c) at (1,-2) {$C$};
        \node[] (cd) at (1,-3) {$CD$};
        % \node[] (d) at (0,-3) {$D$};
        % \node[] (e) at (-1,-3) {$E$};
        \node[] (ef) at (-2,-3) {$EF$};
        % \node[] (f) at (-2,-2) {$F$};
        % \node[] (g) at (-2,-1) {$G$};
        % \node[] (h) at (-2,0) {$H$};

        \graph{
            % (ab) -> (a);
            % (ab) -> (b);
            % (ab) ->[bend left] (c);

            % (cd) -> (c);
            % (cd) -> (d);
            % (cd) ->[bend left] (e);

            % (ef) -> (e);
            % (ef) -> (f);
            % (ef) ->[bend left] (g);
            % (ef) ->[bend left] (h);
            % (ef) -> (a);
        }
    }
\end{center}
Für die rechten Seiten bekommt
jedes Attribute seinen eigenen Knoten.
Hier also $C$, $E$, $A$, $G$, und $H$.
Die Kanten ergeben sich eben aus den FDs.
\begin{center}
    \tikz{
        \node[] (a) at (0,0) {$A$};
        \node[] (ab) at (1,0) {$AB$};
        % \node[] (b) at (1,-1) {$B$};
        \node[] (c) at (1,-2) {$C$};
        \node[] (cd) at (1,-3) {$CD$};
        % \node[] (d) at (0,-3) {$D$};
        \node[] (e) at (-1,-3) {$E$};
        \node[] (ef) at (-2,-3) {$EF$};
        % \node[] (f) at (-2,-2) {$F$};
        \node[] (g) at (-2,-1) {$G$};
        \node[] (h) at (-2,0) {$H$};

        \graph{
            % (ab) -> (a);
            % (ab) -> (b);
            (ab) ->[bend left] (c);

            % (cd) -> (c);
            % (cd) -> (d);
            (cd) ->[bend left] (e);

            % (ef) -> (e);
            % (ef) -> (f);
            (ef) ->[bend left] (g);
            (ef) ->[bend left] (h);
            (ef) -> (a);
        }
    }
\end{center}
Wir sind aber noch nicht fertig.
Wie wir uns überlegt haben,
können wir die Attributmenge
auf der linken Seite mit in die rechte übernehmen.
$$
    A_i \to B_i \implies
    A_i \to A_i, B_i 
$$

Diese Kanten sind im späteren Verlauf
auch wichtig und wir wollen sie auch gleich einzeichnen.
\begin{center}
    \tikz{
        \node[] (a) at (0,0) {$A$};
        \node[] (ab) at (1,0) {$AB$};
        \node[] (b) at (1,-1) {$B$};
        \node[] (c) at (1,-2) {$C$};
        \node[] (cd) at (1,-3) {$CD$};
        \node[] (d) at (0,-3) {$D$};
        \node[] (e) at (-1,-3) {$E$};
        \node[] (ef) at (-2,-3) {$EF$};
        \node[] (f) at (-2,-2) {$F$};
        \node[] (g) at (-2,-1) {$G$};
        \node[] (h) at (-2,0) {$H$};

        \graph{
            (ab) -> (a);
            (ab) -> (b);
            (ab) ->[bend left] (c);

            (cd) -> (c);
            (cd) -> (d);
            (cd) ->[bend left] (e);

            (ef) -> (e);
            (ef) -> (f);
            (ef) ->[bend left] (g);
            (ef) ->[bend left] (h);
            (ef) -> (a);
        }
    }
\end{center}

\end{document}
